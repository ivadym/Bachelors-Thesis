\chapter{Impactos del Trabajo Fin de Grado}

El empleo de las computadoras para la realización de múltiples y variadas tareas aumenta constantemente en la sociedad, por lo que proporcionar a estas máquinas la capacidad de una percepción emocional puede suponer un nuevo punto de inflexión en el campo de inteligencia artificial. De hecho, diseñar un sistema que sea capaz de reconocer fielmente la expresión facial o el estado de ánimo de una persona podría permitir que las computadoras realizaran una gran variedad de nuevas y complejas tareas que en un principio implicaban la necesidad de disponer de una gran comprensión del entorno y de la situación. Algunos ejemplos de estas labores podrían ser la ocupación del cuidado de las personas de la tercera edad, la participación en diversos ejercicios de rehabilitación o simplemente la monitorización de los estados anímicos.

Desde otro punto de vista, al ser las expresiones faciales un reflejo del estado interno de una persona, esta información resulta especialmente valiosa para la obtención de una realimentación directa e instantánea ante un estímulo. Esta cuestión es precisamente la que es cada vez más explotada por las distintas empresas comerciales que buscan mejorar las ventas y los productos que ofrecen.

De hecho, esta aspiración de reunir cada vez una cantidad mayor de datos a partir de los cuales extraer conclusiones es lo que ha detonado y favorecido el espectacular avance de las técnicas de aprendizaje automático y profundo. Sin embargo, estos métodos requieren de una inmensa cantidad de energía, por lo que, a pesar de que el consumo íntegro de Google proviene de fuentes renovables, los impactos medioambientales son evidentes.

Tal y como puede observarse, el reconocimiento de emociones es un área que tiene un gran potencial para ser explotado por múltiples campos heterogéneos, como son los académicos, los sociales, los cínicos o los económicos y comerciales. En el caso particular de este proyecto el área en el que se tiene un mayor impacto es la académica o investigadora, ya que el objetivo principal de este escrito ha sido el diseño de un sistema que sea especialmente competitivo en las tareas del reconocimiento de emociones sobre una base de datos normalizada. En menor medida y en el contexto del estudio de los servicios que podría ofrecer el sistema desarrollado, el módulo, de implantarse en el espejo inteligente final, podría desempeñar una labor de soporte a los profesionales de los sectores médicos. Esto último, por curso natural, también implicaría un impacto económico para el proyecto europeo dentro del cual se engloba la parte de implementación de este trabajo.

\clearpage
\newpage
