\chapter{Introducción}

\section{Motivación}
Las emociones son especialmente importantes en la inteligencia humana, en la toma de decisiones racionales, en la interacción social, en la percepción, en la memoria, en el aprendizaje y en la creatividad. Además, son indispensables para un desarrollo y una gestión inteligente de las situaciones diarias. Por todo ello, la habilidad de reconocerlas de forma precisa y automatizada es una fuente extremadamente valiosa de información, tanto en el ámbito tecnológico, como social y económico. De hecho, esta capacidad de obtener un estado anímico a partir de la observación de una expresión emocional y a través de un razonamiento sobre la situación afectiva que se está dando, es uno de los sellos de identidad de la inteligencia artificial.

Esta detección de las emociones es fundamentalmente necesaria para la obtención de un mejor servicio por parte de las máquinas \cite{Picard}, que dotadas de cierta inteligencia afectiva en este proceso de reconocimiento, se beneficiarán de una toma de decisiones más flexible y racional, de la habilidad de determinar prominencias en el comportamiento humano, dando lugar a una atención y a una percepción más naturales, de la capacidad de abordar múltiples asuntos de una manera inteligente y eficiente, y de otras muchas más interacciones con los procesos cognitivos. Todo ello es realmente útil, por ejemplo, en las áreas en las que son utilizados dispositivos inteligentes para el cuidado de los grupos de la tercera edad, para los tratamientos de inserción social de individuos con autismo, para la rehabilitación de personas con parálisis facial o para atender a los distintos pacientes de un hospital, tareas que exigen una comprensión y un análisis profundo del entorno.

Es evidente, por lo tanto, que para conseguir una respuesta lo más natural y acertada posible por parte de una computadora ante un estímulo, es necesario que ésta logre imitar los procedimientos mediante los cuales el cerebro humano procesa la información sensorial y los razonamientos. En este sentido, por medio de la toma de una serie de datos del entorno y su posterior procesamiento mediante los algoritmos adecuados que imitan el funcionamiento de una red neuronal biológica, es posible, a día de hoy, generar una percepción artificial que iguala o excede, incluso, las capacidades humanas en algunos ámbitos \cite{AIreport}. Estos avances, de hecho, también han sido en gran parte gracias a las novedosas técnicas de aprendizaje profundo y automático que han sido desarrolladas en los últimos años.

En este contexto, y dado que las expresiones faciales de un individuo reflejan su estado interno, es deducible que si un ordenador es capaz de capturar una secuencia de imágenes faciales, entonces el uso de técnicas de aprendizaje profundo nos permitiría conocer el estado de ánimo de su interlocutor. 

En consecuencia, es posible afirmar que estos procedimientos y métodos tienen el potencial de convertirse en un factor clave en el avance de la inteligencia artificial, y por lo tanto, en el progreso y mejora de las habilidades de los ordenadores en su labor de comprensión, interacción y ofrecimiento de soluciones y servicios a los seres humanos.

\section{Objetivos}

El principal objetivo del proyecto de fin de titulación propuesto es el de desarrollar un modelo que sea capaz de reconocer, lo más fielmente posible y acercándose al estado del arte, las emociones básicas y universales (ira, asco, miedo, alegría, tristeza y sorpresa)  \cite{Ekman}, así como la ausencia de éstas (neutral), a partir de imágenes faciales estáticas y etiquetadas conforme a la expresión facial escenificada.

Por otro lado, y exclusivamente como aplicación directa del modelo desarrollado, se pretende implementar un sistema que sea capaz de reconocer las emociones de un individuo en tiempo real y que pueda ser integrado en un sistema empotrado como parte de un prototipo de espejo inteligente.

Finalmente, dado que éste es un proyecto multidisciplinar que involucra distintos ámbitos como la computación afectiva, el aprendizaje automático y profundo y la visión por computador, otro de los objetivos es el de aprender la forma en la que se relacionan estos campos y cómo su confluencia puede proporcionar soluciones a problemas complejos.

\section{Contribuciones}

Mediante el presente escrito se pretende ofrecer un nuevo enfoque dentro del ámbito del reconocimiento de las expresiones faciales, logrando unas tasas de precisión sin precedentes con unos recursos limitados y combinando métodos estándar, como son la utilización de las redes neuronales, la transferencia de aprendizaje, el aprendizaje profundo y la visión por computador.
 
En este sentido, las principales contribuciones de este trabajo pueden englobarse en los siguientes puntos:
\begin{itemize}
  \item Estudio, desarrollo y adaptación a la tarea de identificación de emociones de una serie de modelos preentrenados, y que a día de hoy representan el estado del arte del reconocimiento facial y de imágenes.
  \item Implementación y entrenamiento de estos modelos en la plataforma Google Cloud.
  \item Desarrollo e implementación de un tipo de red generativa antagónica, cuya capacidad de generación de imágenes de forma artificial se pretende aprovechar para extender la base de datos inicial y mejorar, de esta forma, los resultados obtenidos anteriormente.
  \item Desarrollo y despliegue de un sistema de reconocimiento de expresiones faciales en tiempo real como parte de un espejo inteligente en la \textit{Smart House Living Lab} de la E.T.S.I. de Telecomunicación.
\end{itemize}

Por último, hace falta añadir que el código fuente de este trabajo está disponible, junto con los resultados y las instrucciones para replicarlos, en el siguiente repositorio: \url{https://github.com/ivadym/facial-expression-recognition}.