\chapter{Extensión de la Base de Datos FER-2013} \label{Chapter:6}

Oobtenidos unos resultados más que aceptables y eficientes en lo que respecta al tiempo de entrenamiento, se ha intentado dar un paso más allá con la intención de mejorar las tasas de acierto y la calidad de la base de datos inicial mediante la implementación de una Red Generativa Antagónica (GAN) que permita la producción artificial de imágenes y, por lo tanto, la eliminación del desequilibrio de la distribución de clases del conjunto FER-2013.

\section{Arquitectura de la Red Generativa Antagónica de Ciclo Consecuente}

\section{Entrenamiento}

\section{Resultados}