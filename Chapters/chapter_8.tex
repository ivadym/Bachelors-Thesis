\chapter{Conclusiones} \label{Chapter:8}

\section{Conclusiones}

A lo largo de este documento se han investigado y combinado algunas técnicas de visión por computador e inteligencia artificial para clasificar las expresiones faciales. Como se ha podido comprobar, este es un problema complejo que ha requerido de un análisis profundo para obtener unos resultados, que aunque no han sido los mejores, han alcanzado unos valores significativamente competitivos teniendo en cuenta el tiempo y los medios que se han invertido para afinar el modelo ResNet-50 pre-entrenado con la base de datos VGGFace2. En este contexto, la solución aquí planteada al problema del reconocimiento de expresiones faciales puede suponer una gran oportunidad para todos aquellos que deseen implementar un identificador de emociones con una respuesta más que aceptable y dispongan de recursos computacionales limitados.

De forma concreta, en un principio en este trabajo se propusieron dos puntos de vista desde los cuales abordar la tarea de identificaciones de emociones. Por un lado se han utilizado los modelos Inception-v3 e Inception-ResNet-v2 pre-entrenados sobre el conjunto ImageNet, mientras que por otro es empleada la red ResNet-50 preentrenada con la base de datos VGGFace2. A pesar de que objetivamente los dos primeros modelos han reportado mejores resultados en amplias tareas de aprendizaje profundo, el hecho de que ResNet-50 tenga los pesos adaptados al conjunto VGGFace2 marca la diferencia en favor de este último modelo. De hecho, las desigualdades reportadas son significativas, alcanzando el sistema ResNet-50 una precisión de $71.25\%$ sobre el conjunto de evaluación de la base de datos FER-2013, una tasa bastante superior a los $65.00\%$ y $63.86\%$ de las arquitecturas Inception-ResNet-v2 e Inception-v3 respectivamente. Esto evidencia la inmensa importancia que tiene la naturaleza de las imágenes empleadas en los modelos pre-entrenados. En realidad, el número y la calidad de los datos son fundamentales para obtener un buen desempeño, más incluso que el diseño de la propia arquitectura en algunas ocasiones, tal y como se ha podido comprobar a lo largo de este proyecto. Precisamente por este motivo es por el cual se ha decidido explorar la generación artificial de imágenes mediante las redes generativas antagónicas. Sin embargo, puesto que éstas son muy costosas de entrenar, finalmente no se ha podido llegar a unos resultados concluyentes al realizarse tan solo un entrenamiento parcial. A pesar de ello, hay numerosas evidencias empíricas que muestran que la aplicación de éstas técnicas de generación de datos pueden aumentar entre el 5\% y el 10\% el desempeño de los modelos iniciales \cite{GANAugmentation}, lo que en nuestro caso supondría la superación del estado del arte actual.

En última instancia también se ha logrado implementar un modelo de reconocimiento de expresiones faciales en tiempo real en un sistema empotrado, explorándose los servicios que este desarrollo sería capaz de ofrecer desde una perspectiva biomédica.

Finalmente, hace falta añadir que el código fuente de este trabajo está disponible, junto con los modelos entrenados, los resultados y las instrucciones para replicarlos, en el siguiente repositorio: \url{https://github.com/ivadym/FER}.

\section{Líneas Futuras}

Dada la multidisciplinariedad de este proyecto, hay una gran variedad de vías por las que seguir las investigaciones aquí plasmadas. De esta forma, por un lado sería interesante continuar con el objetivo inicial de diseño de un sistema de reconocimiento de expresiones faciales óptimo. En este contexto, podría resultar beneficioso realizar un entrenamiento de la red ResNet-50 completa y con una tasa de aprendizaje aún menor con el objetivo de obtener una mejor convergencia. Asimismo, también valdría la pena intentar terminar el entrenamiento incompleto de las redes CycleGAN, generar nuevas imágenes para la categoría correspondiente a la expresión de asco y comprobar si las mejoras que se obtienen son significativas. En caso de que lo fueran, es probable que la generación de más datos de las clases menos representadas pudiera dar lugar a que se superase el estado del arte actual.

Por otro lado y más en un contexto práctico, se podría enfocar la continuación de este proyecto a la obtención de un sistema de reconocimiento más heterogéneo, combinando varias bases de datos de expresiones faciales distintas. Esto tendría como objetivo mejorar el desempeño del reconocimiento en situaciones reales, en lugar de centrarse en obtener las mayores tasas sobre una base de datos determinada.

Esta última perspectiva es precisamente la que permitiría obtener un mejor funcionamiento del sistema implantado en el prototipo de espejo inteligente. Por ello, ésta es otra de las alternativas en las que se pueden enfocar los trabajos futuros: optimizar y añadir funcionalidades al espejo inteligente dentro del proyecto HOOP anteriormente descrito.
