\thispagestyle{plain}

\begin{summary}
Facial expression recognition is currently a very active topic in the fields of computer vision and artificial intelligence, being exploited in areas as diverse as academic, clinical or commercial.  However, the emotion recognition is not an easy problem for computers, and sometimes not for humans as well, since people can vary significantly in the way that they show their expressions. In this context, providing machines with the ability to identify mood can significantly improve their performance in a wide variety of tasks, even in some very complex that require high emotional intelligence.

To address these problems and even though the emotion recognition can be conducted using multiple sensors, this project focuses exclusively on the study of facial images as they are one of the main information channels in interpersonal communication. In this way, this document provides a brief review of researches in the field of computer vision and artificial intelligence and proposes a facial expression recognition system that uses convolutional neural networks and the transfer learning technique. As a result, various widely extended convolutional architectures are explored (Inception-v3, Inception-ResNet-v2 and ResNet-50), as well as several databases of different nature (ImageNet, VGGFace2 and FER-2013). Likewise, an analysis and implementation of some data augmentation methods are also carried out, both from the point of view of the image preprocessing and from an artificial data generation approach through generative adversarial networks. Besides, due to the complexity of the models implemented, the computational resources offered by the Google Cloud platform are used in order to reduce the training time.

Ultimately, the proposed approach has shown to be very effective, improving a large part of the results reported to date on the database FER-2013 with a minimal computational and temporal investment. Additionally, this approach has also favored the implementation of the facial expression model in an embedded system, enabling a wide range of services that could be offered in pseudo real time.
\end{summary}

\begin{keywords}
Artificial Intelligence, Computer Vision, Deep Learning, Facial Expression Recognition, Convolutional Neural Networks, Transfer Learning, Cycle-Consistent Adversarial Networks, Google Cloud, FER-2013.
\end{keywords}