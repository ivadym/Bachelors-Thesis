\thispagestyle{plain}

\begin{resumen}
El reconocimiento de expresiones faciales es un campo de estudio muy activo actualmente en las áreas de visión e inteligencia artificial, abarcando ámbitos tan diversos como los académicos, los clínicos o los comerciales. Sin embargo, este reconocimiento de emociones no es un problema sencillo para las computadoras, y en algunas ocasiones tampoco para los humanos, ya que cada individuo puede manifestar su estado afectivo de una manera distinta. En este contexto, proporcionarles a las máquinas la capacidad de identificar la condición anímica de una persona puede favorecer significativamente su desempeño en una gran variedad de tareas, incluso en algunas muy complejas que requieren de una elevada inteligencia emocional.

Con el fin de abordar estos problemas y aunque el reconocimiento de emociones puede realizarse utilizando múltiples sensores, este proyecto se centra exclusivamente en el estudio de las imágenes faciales al ser éstas uno de los principales canales de información en una comunicación interpersonal. De esta manera, este documento proporciona una breve reseña de las investigaciones en el campo de la visión e inteligencia artificial y propone un sistema de reconocimiento de expresiones faciales basado en redes neuronales convolucionales y en la técnica de transferencia de aprendizaje. A raíz de ello, son exploradas diversas arquitecturas convolucionales ampliamente extendidas (Inception-v3, Inception-ResNet-v2 y ResNet-50), así como varias bases de datos de diferente índole (ImageNet, VGGFace2 y FER-2013). Asimismo, también son llevados a cabo un análisis y una implementación de algunos métodos de aumento de datos, tanto desde el punto de vista del preprocesamiento de imágenes como desde un enfoque de generación de representaciones artificiales mediante redes generativas antagónicas. Por último, dada la complejidad del problema planteado y por consiguiente del sistema desarrollado para resolverlo, son aprovechados los recursos computacionales que proporciona la plataforma Google Cloud para disminuir el coste temporal del entrenamiento de los modelos desarrollados.

En definitiva, el planteamiento propuesto en este escrito ha demostrado ser muy efectivo, mejorando una gran parte de los resultados reportados hasta la fecha sobre el conjunto FER-2013 y además, con una inversión computacional y temporal mínima. Adicionalmente, este enfoque también ha permitido implementar el modelo del reconocimiento de expresiones faciales desarrollado en un sistema empotrado, abriéndose un amplio abanico de servicios que podrían ofrecerse en pseudo tiempo real.
\end{resumen}

\begin{palabrasclave}
Inteligencia Artificial, Visión por Computador, Aprendizaje Profundo, Reconocimiento de Expresiones Faciales, Redes Neuronales Convolucionales, Transferencia de Aprendizaje, Redes Generativas Antagónicas de Ciclo Consecuente, Google Cloud, FER-2013.
\end{palabrasclave}
